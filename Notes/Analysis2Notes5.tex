\begin{prop}
  If any function $f:\Set{U}\subset\RR^n\to\RR^m$, for $\Set{U}$ open is differentiable in $x_0\in\Set{U}$, it is also continuous in $x_0$. 
\end{prop}
\begin{proof}
  By differentiability in $x_0$, $\exists D$ s.t. $$f(x)=f(x_0)+Df(x_0)(x-x_0)+\phi(x-x_0)$$ with $lim_{x\to x_0}\frac{\phi(x-x_0)}{\Norm{x-x_0}}$. Thus $\lim_{x\to x_0}\phi(x-x_0)=0$, so $\lim_{x\to x_0}f(x)=f(x_0)+Df(x_0)(x-x_0)=f(x_0)$.
\end{proof}


\begin{prop}[linearity]
  Let $f, g:\Set{U}\to \RR^m$ differentiable in $x_0\in \Set{U}, \alpha, \beta\in\RR$, $\Set{U}\subset\RR^n$ open, then $\alpha f+\beta g:\Set{U}\to\RR^m$ is differentiable in $x_0$ and $D(\alpha f(x_0)+\beta g(x_0)=\alpha Df(x_0)+\beta D g(x_0)$.
\end{prop}
\begin{proof}
  Plugging in the definition of differentiability, we yield:
  \begin{align*}
  	&\lim_{x\to x_0}f(\alpha x+\beta x)\\
  	=&\lim_{x\to x_0}f(\alpha x_0+\beta x_0)+Df(\alpha x+\beta x-\alpha x_0-\beta x_0)+\phi(\alpha x+\beta x-\alpha x_0-\beta x_0)\\\overset{\phi\to 0}{=}&\lim_{x\to x_0}f(\alpha x_0+\beta x_0)+Df(\alpha x+\beta x-\alpha x_0-\beta x_0)\\&\text{By definition of differentiability $Df$ is linear:}\\
  	\Rightarrow& Df(\alpha x+\beta x-\alpha x_0-\beta x_0)=\alpha Df(x_0)+\beta D g(x_0)
  \end{align*}
\end{proof}


\begin{thm}[chain rule]
  Let $f:\Set{U}\to\RR^m, \Set{U}\in\RR^n$ open be differentiable in $x_0\in \Set{U}$ and $g:\Set{V}\to RR^p$, for $\Set{V}\in \RR^n$ open with $y_0:=f(x_0)\in \Set{V}$. Then $g\circ f: \Set{U}\to \RR^n$ is differentiable in $x_0$ and $D(g\circ f)(x_0)=Dg(f(x_0))\circ Df(x_0)$.
\end{thm}
\begin{proof}
  From differentiable of $f$ in $x_0$ and $g$ in $y_0$, we obtain: 
  \begin{equation}
  	\label{eqn:5.1}
  	f(x)=f(x_0)+Df(x_0)(x-x_0)+\phi(x-x_0), \lim_{x\to x_0}\frac{\phi(x-x_0)}{\Norm{x-x_0}}
  \end{equation}
  \begin{equation}
  \label{eqn:5.2}
  f(y)=f(y_0)+Df(y_0)(y-y_0)+\phi(y-y_0), \lim_{y\to y_0}\frac{\phi(y-y_0)}{\Norm{y-y_0}}
  \end{equation}
  Putting $y:=f(x), x\in\Set{U}$, we get (allowed since $f$ continuous):
  \begin{equation}
  \label{eqn:5.3}
  y-y_0=f(x)-f(x_0)=Df(x_0)(x-x_0)+\phi(x-x_0)
  \end{equation}
  Using this equations, we wan to give a linear approximation of $g\circ f$ in $g(f(x_0))=g(y_0)$: 
  \begin{align*}
  	&g(f(x))\\
  	\eqSince{\eqref{eqn:5.2}}&g(f(x_0))+\left(Dg(f(x_0)(x-x_0)+\phi(x-x_0)\right)+\phi(f(x)-f(x_0))\\
  	\eqSince{\eqref{eqn:5.1}}&g(f(x_0))+Dg(f(x_0))\left(Df(x_0)(x-x_0)+\phi(x-x_0)\right)+\phi(f(x)-f(x_0))\\
  	\eqSince{linearity}&g(f(x_0))+Dg(f(x_0))\left(Df(x_0)(x-x_0)\right)+Dg f(x_0)\phi(x-x_0)+\psi(f(x_0)-f(x_0)). \\ &\text{\hspace{-1cm}Defining $\psi(\theta(fx-x_0):=\psi(f(x)-f(x_0))+Dg(f(x_0))\phi(x-x_0)$:}\\
  	=&(g\circ f)(x_0)+Df(x_0)\circ D(g\circ f)(x_0)(x-x_0)+\theta(x-x_0).
  \end{align*}
  Firstly, since $Dg(f(x_0))\in L(\RR^m,\RR^n)$, it follows from \[\lim_{x\to x_0}\frac{\phi(x-x_0)}{\Norm{x-x_0}}=0, \text{ that}\]
  \[\lim_{x\to x_0}\frac{Dg(f(x_0))\phi(x-x_0)}{\Norm{x-x_0}}=0\]
  Secondly, define $\phi_0$ by $\phi(y-y_0)=:\Norm{y-y_0}\Psi_0(y-y_0)$, so $\lim_{x-x_0}\Psi_0(y-y_0)=0$ and use \eqref{eqn:5.1}, the triangular inequality and the operator norm to estimate:
  \begin{align*}
  	&\Norm{\Psi(f(x)-f(x_0)}=\Norm{f(x)-f(x_0)}\Norm{\Psi_0(f(x)-f(x_0)}=\Norm{df(x_0)(x-x_0)+\phi(x-x_0)}\\
  	=&\Norm{\Psi_0(f(x))-f(x_0)}\leq\left(\Norm{Df(x_0)(x-x_0)}+\Norm{\phi(x-x_0)}\right)\Norm{\Psi_0(f(x)-f(x_0)}\\
  	=&\Norm{Df(x_0)}\Norm{x-x_0}
  \end{align*}
  Division by $\Norm{x-x_0}$ yields:
  \begin{align*}
  	0&\\\leq&\frac{\Psi(f(x)-f(x_0)}{\Norm{x-x_0}}\\
  	\leq&\left(\Norm{Df(x_0)}+\frac{\Norm{\phi(x-x_0)}}{x-x_0}\right)\Norm{\Psi(f(x)-f(x_0)}
  \end{align*}
  Since $\lim_{z\to 0}\Psi_0(z)=0$ and $f$ is continuous in $x_0$, so $\lim_{x\to x_0}f(x)=f(x_0)$, we obtain $\lim_{x\to x_0}\Psi_0(f(x)-f(x_0))=0$. We can conclude \[\lim_{x\to x_0}\text{RHS}=0\Rightarrow\lim_{x\to x_0}\frac{\Psi(f(x)-f(x_0)}{\Norm{x-x_0}}=0.\]
\end{proof}
\begin{prop}[Lipschitz continuity and $Df=0\Rightarrow f$ constant]
  Let $f:\Set{U}\to \RR^m$, $\Set{U}\subset\RR^n$ open and convex, i.e. $\forall u, v\in\Set{U}, \forall t\in[0,1]:\:(1-t)u+tv\in\Set{U}$, be differentiable in $\Set{U}$ s.t. $\exists M\geq 0$ with $\Norm{Df(x)}\leq M, \forall x\in\Set{U}$. Then, $f$ is $M$-Lipschitz continuous on $\Set{U}$, i.e. $\Norm{f(x)-f(y)}\leq M\Norm{x-y}, \forall x, y\in\Set{U}$. In particular, if $M=0$ meaning that $Df$ is constant on $\Set{U}$ then $f$ is a constant function.
\end{prop}
\begin{proof}
  To be continued\ldots
\end{proof}
\begin{rem}
  The special case is a generalisation of hte statement that if $f:I\to \RR$ is differentiable with $f\cong 0$, then $f$ must be constant. 
\end{rem}
\begin{defn}
  Let $f:\Set{U}\to\RR, \Set{U}\subset\RR^n$ open and consider $x_0\in\Set{U}$ and $v\in\RR^n$ so that $\Norm{v}=1$. If the limit exists $D_vf(x_0):=\lim_{t\to 0}\frac{f(x_0+tv)-f(x_0)}{t}$, it is called the \emph{directional derivative} of $f$ in $x_0$ in direction of $v$. 
\end{defn}
\begin{exam}
  $f:\RR^2\to \RR, f(x,y):=\begin{cases}
  \frac{xy}{x^2+y^2}\qquad (x,y)\neq 0\\0\qquad\text{otherwise}\end{cases}$. We can check that the partial and even the total differential exist in all $(x,y)\neq 0$, but $f$ is not continuous in $0$. It is, however, more interesting to look for $f$ in $0$. So we will later come back to this example. 
\end{exam}
\begin{defn}
  If $v=e_j\qquad\forall j\in\upto{n}$ in previous definition, then the corresponding directional derivatives are called the \emph{partial derivatives with respect to $x_j$} of $f$ in $x_0$ denoted by $f_{x_j}(x_0), D_j(f(x_0)), \frac{\partial f}{\partial x_j}(x_0)$. 
\end{defn}
\begin{exam}\ 
  \begin{enumerate}
  	\item $f:\RR\to\RR, f(x,y):=x^2+y^2$, $\frac{\partial f}{\partial x}((x,y))=2x, \frac{\partial f}{\partial y}((x,y))=2y$
  	\item $g:\RR^2\to\RR, g(x,y):=\begin{cases}
  	\frac{xy}{x^2+y^2}\qquad (x,y)\neq 0\\0\qquad\text{otherwise}\end{cases}$. $\frac{\partial g}{\partial x}((0,0))=\lim_{t\to 0}\frac{g((t,0))}{t}=\lim_{t\to 0}\frac{0}{t}=0$, $\frac{\partial g}{\partial y}((0,0))=\lim_{t\to 0}\frac{g((0,t))}{t}=\lim_{t\to 0}\frac{0}{t}=0$. However, $g$ is not continuous in 0 let alone differentiable: $y:=\alpha x_i, \lim_{x\to 0}\frac{x\alpha x}{x^2+\alpha^2x^2}=\frac{\alpha}{1+\alpha^2}$
  	\item Show that $f$ has partial derivatives everywhere apart from the origin. 
  \end{enumerate}
\end{exam}
\begin{defn}
  $f:\Set{U}\to\RR$, $\Set{U\subset\RR}$ open is called \emph{partially differentiable} iff partial derivatives exist and are continuous on $\Set{U}$. 
\end{defn}
\begin{exam}
  \begin{enumerate}
  	\item $f:\RR^2\to\RR, f((x,y)):=x^2+y^2$, is continuously partially differentiable on $\RR^2$, since both $\RR^2\to \RR, (x,y)\to2x, (x,y)\to2y$ are continuous on $\RR^2$. 
  	\item Norms on $\RR^k$ are continuously partially differentiable on $\RR^k\setminus\{0\}$. 
  \end{enumerate}
\end{exam}
\begin{defn}[Jacobean]
  $f:\Set{U}\to\RR^m$, $\Set{U}\subset\RR^n$ open, so that all partial derivatives $\frac{\partial f_i}{\partial x_j}(x_0), \forall i\in\upto{m}, j\in\upto{n}$ in $\x_o$ exist. Then the matrix $J_f(x_0):=\left(\frac{\partial f_i}{\partial x_j}\right)_{1\leq n}^{1\leq m}\in\RLinSpaNM{m}{n}$ is called the \emph{Jacobean matrix} of $f$ in $x_0$. 
\end{defn}
\begin{rem}
  We will see that in the special case $m=1$, the Jacobean is also called gradient of $f$ in $x_0$ and has a special geometric meaning. 
\end{rem}
\begin{exam}
  $f:\RR^3\to\RR^2, f((x,y)):=\lrrvector{x^2+y^2+z^2}{xyz}$. $f$ is continuously partially differentiable everywhere and the Jacobean is given by $J_f((x,y,z))=\lrrrccmatrix{2x}{2y}{2z}{yz}{xz}{xy}\in\RLinSpaNM{2}{3}$. 
\end{exam}
