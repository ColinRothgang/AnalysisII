\section{Taylor's formula in several variables}
\subsubsection*{Idea}
We shall first generalize the Taylor in one variable: 
\begin{equation*}
  f(x+h) = \sum_{i=0}^k\frac{f^{(i)}(x)}{i!}h^i + \frac{f^{(k+1)}(\theta)}{(k+1)!} \:\:\: \theta \in [x, x+h]
\end{equation*}
Differentiability insures the existence of a linear approximation of the function of several variables; with higher differentiability resulting in ``higher order approximations''.

\subsubsection*{Notation:}
For $\alpha = (\alpha_1, \dots, \alpha_n) \in (\NN \union {0})^n$
$|\alpha| := \alpha_1 + \dots + \alpha_n$
$\alpha ! := \alpha_1! \times \dots \times \alpha_2!$
For $x = (x_1, \dots, x_n) \in \RR^n$
$x^\alpha := x_1^{\alpha_1} \times \dots \times x_n^{\alpha_n}$

\begin{thm}[Taylor's formula in several variables]
Let $f: u \toward \RR, u \subset \RR^n$ (open) be $p$ times partially continuously differentiable, i.e:
\begin{equation*}
  D^\alpha = D_1^{\alpha_1}\dots D_n^{\alpha_n}f = \frac{\partDeriv{^{|\alpha|}f}}{\partDeriv{x_1^{\alpha_1}}\dots \partDeriv{x_n^{\alpha_n}}} = \frac{\partDeriv{^{\alpha_1}}}{\partDeriv{x_1^{\alpha_1}}}(\frac{\partDeriv{^{\alpha_n}}}{\partDeriv{x_{n-1}^{\alpha_{n-1}}}}(\dots))
\end{equation*}
exists and are continuous $\forall \alpha \in (\NN \union {0})^n$ with $|\alpha| \leq p$. Assume $x\in u$ and $h \in \RR^n$ so that $[x, x+h] \subset u$. $\exists \theta \in [0, 1]$ so that
\begin{equation*}
  f(x+h) = \sum_{|\alpha| \leq k}\frac{D^\alpha f(x)}{\alpha !}h^\alpha + \sum_{|\alpha| = k+1}\frac{D^\alpha f(x+\theta h)}{\alpha !} h^\alpha
\end{equation}



\begin{thm}[Generalized Schwarts theorem]
  \being{equation*}
  
\end{equation*}

\end{thm}

\subsubsection*{Motivation}
Let $F$ be a differentiable function in $x, y\in\RR$. Given $F(x,y)=0$, on can ask, when there is a differentiable (locally defined) function $g(x)=y$, s.t. $F(x,g(x))=0$. $F(x,y)=0$ could for instance describe a level set. 

We need that $\partDeriv{F}{y}(x,y)\neq 0$, then (in $\RR$) $$g'(x_0)=-\frac{\partDeriv{F}{x}}{\partDeriv{F}{y}}.$$
\begin{exam}
  $F(x,y)=0$, for $F:\RR^2\to\RR$, $F(x,y)=x^2+y^2-r^2\Rightarrow \partDeriv{F}{x}=2x, \partDeriv{F}{y}=2y, \partDeriv{F}{y}=0\Leftrightarrow x=\pm r$, thus for $x\neq \pm r$, we obtain $g'(x)=-\frac{x}{y}$.
\end{exam}
Before stating the general implicit function theorem, let us frst consider a linearized version. 
\subsubsection*{Notation:}
For $\RR^m\ni x=(\fseq{x}{m}), \RR^n\ni x=(\fseq{y}{n})$, we write $(x,y):=(\fseq{x}{m},\fseq{y}{n})\in\RR{m+n}$. For $A\in L(\RR^{n+m},\RR^m)$, we define $A_x\in L(\RR^n, \RR^m)$ and $Ay\in L(\RR^m)$, by $\begin{cases}A_x(u):=A(u,0)\qquad\forall u\in\RR^n\\A_y(v):=A(0,v)\qquad\forall v\in\RR^m\end{cases}$. Thus, $A(u,v)=A_x(u)+A_y(v)\qquad\forall(u,v)\in\RR{n+m}$.

\begin{thm}
  Let $A\in L(\RR^n, \RR^m)$ be a linear map s.t. $A_y\in \Set{I}(\RR^m)$, i.e. is invertible. Then $\forall u\in\RR^n.\existsUnique v\in\RR^m$ so that $A(u,v)=0$, namely $v:=-(A_y^){-1}\circ A_x(u)$. 
\end{thm}
\begin{proof}
  \begin{align*}
  	&A(u,v)=0\\\Leftrightarrow &A_x u+A_y v=0\\\Leftrightarrow &A_y^{-1}A_xu+A_y^{-1}A_y v=0\\\lequiv{$A_y\in \Set{I}(\RR^m)$}&A_y^{-1}A_xu+v=0\\\Leftrightarrow &v=-A_y^{-1}A_x u.
  \end{align*}
\end{proof}

\begin{thm}[implicit function theorem]
  \label{thm:impFun}
  Let $F:\Set{U}\to\RR^m, \Set{U}\subset \RR{n+m}$ open be continuously differentiable s.t. $F(x_0,y_0)=0$, for some $x_0, y_0\in\Set{U}$, $x_0\in\RR^n, y_0\in\RR^m$, where $A:=D f(x_0,y_0)$ and $A_y\in\Set{I}(\RR^m)$. Then $\exists x\in \Set{V}_0$ neighborhood of $x_0$, $\exists \Set{U}_0\subset \Set{U}$ neighborhood of $(x_0,y_0)$, so that $\forall x_{\Set{V}_0}\existsUnique y\in \RR^m:(x,y)\in\Set{U}_0\land F(x,y)=0$. If $g(x)$ denotes this $y$ by, $g:\Set{V}_0\to\RR^m$ is continuously differentiable and satisfies $F(x,g(x))=0$ and $Dg(x_0)=-A_y^{-1}\circ A_x$.  
\end{thm}
