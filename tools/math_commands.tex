\usepackage{amsmath,amssymb,amsthm,enumerate}
\usepackage{marvosym}
% Define some theorem kinds I frequently need. Have them share the same
%counter and number them per-section for readability.
\newcounter{lecture}
\newcounter{excC}
\newtheorem{thm}{Theorem}[lecture]
\newtheorem*{thm*}{Theorem}
\newtheorem{defn}[thm]{Definition}
\newtheorem{ldefn}{Local definition}[thm]
\newtheorem*{ldefn*}{Local definition}
\newtheorem{lem}[thm]{Lemma}
\newtheorem{prob}[thm]{Problem}
\newtheorem{Satz}[thm]{Satz}
\newtheorem{prop}[thm]{Proposition}
\newtheorem*{Beh}{Behauptung}
\newtheorem{cor}[thm]{Corollary}
\newtheorem{Kor}[thm]{Korollar}
\newtheorem*{rem}{Remark}
\newtheorem*{Anm}{Anmerkung}
\newtheorem{Aufgabe}[thm]{Aufgabe}
\newtheorem{exc}[excC]{Exercise}
\newtheorem*{exc*}{Exercise}
\newtheorem*{hint}{Hints}
\newtheorem*{sol}{Solution}
\newtheorem{exam}[thm]{Example}
\newtheorem{countEx}[thm]{Counter-Example}
\newtheorem*{countEx*}{Counter-Example}
\newtheorem{con}[thm]{Conjecture}
\newtheorem{Ver}[thm]{Vermutung}
% Declaring some math operators I needed at some point.
\DeclareMathOperator{\Bil}{Bil}
\DeclareMathOperator{\Fix}{Fix}
\DeclareMathOperator{\Hom}{Hom}
\DeclareMathOperator{\Mat}{Mat}
\DeclareMathOperator{\ggT}{ggT}
\DeclareMathOperator{\kgV}{kgV}
\DeclareMathOperator{\id}{id}
\DeclareMathOperator{\lcm}{lcm}
\DeclareMathOperator{\Tr}{Tr}
\DeclareMathOperator{\sgn}{sgn}
\DeclareMathOperator{\sign}{sign}
\DeclareMathOperator{\diam}{diam}
\usepackage{mathtools}
%\DeclarePairedDelimiter{\abs}{\lvert}{\rvert}
\DeclarePairedDelimiter{\ceil}{\lceil}{\rceil}
\DeclarePairedDelimiter{\floor}{\lfloor}{\rfloor}
\DeclarePairedDelimiter{\Pairedsupnorm}{\lvert\lvert}{\rvert\rvert}
% My own commands for convenience, mostly mathematics.
\newcommand{\abs}[1]{\left\lvert #1\right\rvert}
\newcommand{\ABS}{\abs{\,}}
\newcommand{\card}[1]{\lvert#1\rvert}
\newcommand{\supNorm}[1]{\Pairedsupnorm{#1}}
\newcommand{\Norm}[1]{\left\lvert\left\lvert#1\right\lvert\right\lvert}
\newcommand{\CC}{\mathbb{C}}
\newcommand{\DD}{\mathbb{D}}
\newcommand{\HH}{\mathbb{H}}
\newcommand{\NN}{\mathbb{N}}
\newcommand{\PP}{\mathbb{P}}
\newcommand{\QQ}{\mathbb{Q}}
\newcommand{\RR}{\mathbb{R}}
\newcommand{\TT}{\mathbb{T}}
\newcommand{\FF}{\mathbb{F}}
\newcommand{\ZZ}{\mathbb{Z}}
\newcommand{\Q}{\mathcal{Q}}
\newcommand{\R}{\mathcal{R}}
\newcommand{\T}{\mathcal{T}}
\newcommand{\Z}{\mathcal{Z}}
\newcommand{\st}{\colon}
\newcommand{\dt}{\, \mathrm{d}t}
\newcommand{\dx}{\, \mathrm{d}x}
\newcommand{\dy}{\, \mathrm{d}y}
\newcommand{\dz}{\, \mathrm{d}z}
\newcommand{\nextIdea}{\Large{$\curvearrowright$}\normalsize\ }
\newcommand{\nextlecture}{\hfill\addtocounter{lecture}{1}\subsection{Lecture \arabic{lecture}}}
\newcommand{\inputNextLecture}[1]{\nextlecture\input{#1\arabic{lecture}}}
\newcommand{\inPro}[2]{\left\langle #1\middle|#2\right\rangle}
\newcommand{\Ball}[2]{\mathrm{B}(#1, #2)}
\newcommand{\lev}[2]{\Set{N}_{#1}\left(#2\right)}

% Some math macros for convienience
%\newcommand{\dot}[1]{\overset{.}{#1}}
\newcommand{\exe}{\begin{exc}\end{exc}}
\renewcommand{\card}[1]{\langle #1\rangle}
\newcommand{\excN}[1]{\begin{exc}[#1]\end{exc}}
\newcommand{\dotdot}[1]{\overset{..}{#1}}
\newcommand{\ran}{\mathrm{Img}}
\newcommand{\Span}[1]{\mathrm{span}\{#1\}}
\newcommand{\img}[1]{\mathrm{Im}(#1)}
\newcommand{\Invert}[1]{\Set{I}\left(#1\right)}
\renewcommand{\phi}{\varphi}
\newcommand{\nin}{\notin}
\newcommand{\nequiv}{\not\equiv}
\newcommand{\opint}[1]{\left]#1\right[}
\newcommand{\Perm}{Perm}
\newcommand{\dd}[1]{\, \mathrm{d}#1}
\newcommand{\dist}[2]{\, \mathrm{dist}(#1,#2)}
\newcommand{\pdis}[3]{\, \mathrm{dist}_{#1}\left(#2,#3\right)}
\newcommand{\partDeriv}[2]{\frac{\partial #1}{\partial #2}}
\newcommand{\angleBetween}[2]{\angle\left(#1,#2\right)}
\newcommand{\dis}{\, \mathrm{dist}}
\newcommand{\set}[1]{\mathcal{#1}}
\newcommand{\Set}[1]{\set{#1}}
\newcommand{\bra}[1]{\left[#1\right]}
\newcommand{\field}[3]{(#1,#2,#3)}
\newcommand{\fieldS}[1]{\field{#1}{+}{\cdot}}
\newcommand{\ordSet}[2]{(#1,#2)}
\newcommand{\ordS}[1]{\ordSet{#1}{<}}
\newcommand{\ordField}[4]{(#1,#2,#3, #4)}
\newcommand{\oF}[1]{\ordField{#1}{+}{\cdot}{<}}

\newcommand{\x}{\times}
\newcommand{\cont}{{\large \Lightning}}
\newcommand{\follows}[1]{\overset{{\text{#1}}}{\implies}}
\newcommand{\since}[1]{\overset{{\text{#1}}}{\Longleftarrow}}
\newcommand{\lequiv}[1]{\overset{{\text{#1}}}{\iff}}
\newcommand{\eqSince}[1]{\overset{{\text{#1}}}{=}}
\newcommand{\leqSince}[1]{\overset{{\text{#1}}}{\leq}}
\newcommand{\geqSince}[1]{\overset{{\text{#1}}}{\geq}}
\newcommand{\existsUnique}{\exists^{1}}

\newcommand{\upto}[1]{\{1,2,\ldots,#1\}}
\newcommand{\fseq}[2]{#1_1, #1_2, \ldots, #1_{#2}}
\newcommand{\rrvector}[2]{\matr{#1\\#2}}
\newcommand{\rrrvector}[3]{\matr{#1\\#2\\#3}}
\newcommand{\lrrvector}[2]{\lmatrix{#1\\#2}}
\newcommand{\lrrrvector}[3]{\lmatrix{#1\\#2\\#3}}
\newcommand{\lrrrrvector}[4]{\lmatrix{#1\\#2\\#3\\#4}}
\newcommand{\lrrrsvector}[1]{\lmatrix{#1_1\\#1_2\\#1_3}}

\newcommand{\lccvector}[2]{\lmatrix{#1, #2}}
\newcommand{\lcccvector}[3]{\lmatrix{#1, #2, #3}}
\newcommand{\lcccsvector}[1]{\lmatrix{#1_1, #1_1, #1_1}}
\newcommand{\ccvector}[2]{\matr{#1, #2}}
\newcommand{\cccvector}[3]{\matr{#1, #2, #3}}

\newenvironment{lematrix}{\left(\begin{matrix}}{\end{matrix}\right)}
\newcommand{\lmatrix}[1]{\begin{lematrix}#1\end{lematrix}}
\newcommand{\matr}[1]{\begin{matrix}#1\end{matrix}}
\newcommand{\lttmatrix}[4]{\lmatrix{#1&#2\\#3&#4}}
\newcommand{\lrrcccmatrix}[6]{\lmatrix{#1&#2&#3\\#4&#5&#6}}
\newcommand{\lrrrccmatrix}[6]{\lmatrix{#1&#2&\\#3&#4\\#5&#6}}
\newcommand{\ltttmatrix}[9]{\lmatrix{#1&#2&#3\\#4&#5&#6\\#7&#8&#9}}
\newcommand{\glttmatrix}[1]{\lttmatrix{#1_{11}}{#1_{12}}{#1_{21}}{#1_{22}}}
\newcommand{\gltttmatrix}[1]{\ltttmatrix{#1_{11}}{#1_{12}}{#1_{13}}{#1_{21}}{#1_{22}}{#1_{23}}{#1_{31}}{#1_{32}}{#1_{33}}}

\newcommand{\Eig}{\mathrm{Eig}}
\newcommand{\diag}{\mathrm{diag}}
\newcommand{\Matrix}[1]{\underline{\underline{#1}}}
\newcommand{\MatrixS}[1]{\Matrix{\Set{#1}}}
\newcommand{\LinFunSpaNM}[2]{\Set{M}^{#1\times #2}}
\newcommand{\RLinSpaNM}[2]{\Set{M}^{#1\times #2}(\RR)}
\newcommand{\CLinSpaNM}[2]{\Set{M}^{#1\times #2}(\CC)}
